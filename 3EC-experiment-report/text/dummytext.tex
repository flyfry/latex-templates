% Copyright (C) 2021, flyfry.

% **This file is not distributed under MIT license.**
% DO NOT redistribute without permission.

弊学科では, 実験が必修科目となっていて, 情報通信に関する様々な知識・事柄を手を動かすことにより習得していく.
本科目においては, 実験そのものは勿論のこと, その報告書についても履修者自身が作成することにより, 実験に関する知識理解をさらに深めていくこととなる.
この報告書の作成において, 高校までおよび大学に入ってからのコンピュータリテラシー教育や, 世間一般においても文書作成のデファクトスタンダードとなっている影響からMicrosoft Wordを使用されている学生が多いものと考えられるが,
ここではその代替手段として \LaTeX を用いた報告書の作成を行うものとする.

\LaTeX は非常に有用であるが, WordなどのWYSIWYG (What You See Is What You Get) と異なり,
その編集はHTMLのように文章構造を記述するスクリプトファイルを編集することにより行うため, Wordと比較すると, やや「とっつきづらさ」を感じることであろう.
しかし, Wordでは多機能すぎるが故に知れば知るほど, どこで何ができるのかわからなくなりつつもあり, GUIメニューへのアクセスが煩雑なことも相まって, 機能の存在を知っていても, 実際に活用されることは稀なものが多い.
さらに, 文書構造と装飾が混在することにより, 先の機能アクセスへの煩雑さも相まって, 文書構造を頭では把握していても, Wordファイルにその情報を埋め込むのではなく, 節番号などを手動で入力して,
太字にして, フォントを大きくして,... と装飾に関する各種作業を完全に手動で行っているものも少なくないのではなかろうか.
無論, 一度きりの使い捨てのメモ書きや配布物など, そうした手法が必ずしも悪であるとは言い難いが, 大学のレポートなど, 修正加筆を繰り返しながら仕上げていくような本格的な「文書」の作成において,
文書構造の指定されていないファイルの修正のためには, 少し文章に手を加えたらスペースで文字位置を修正して... などといった, 本質とは関係のない部分において無用な努力を課されることになることも多いと考えられる.
\footnote{
    ここまでボロクソ書いておいてアレだが, 筆者は別にWordアンチではない. 修正されることを前提としていない装飾だけで書かれたファイルが(個人的に)嫌いなだけである.
    ただし本節にも書いたように, そうした使い方が必ずしも悪ではないため, 必要に応じて使い分けることが重要であろう. レポートなどではご法度である, と言う話である.
}

一方, \LaTeX では文書の構造を明確に記述し, 装飾と分離して考えることが原則である.
組版にあたり見た目に関する設定は, jsarticle などの文書クラスを利用すれば, 特に設定などせずとも ``\textbackslash section\{節題\}'' と記述するだけで, それなりの表示結果が得られる.
このほか, 非常に強力な数式記述が可能であり, 図表の相互参照等も容易であるため, 長大で論理的な文章の記述を強いられる我々理系学生にとって, これを活用しない手はないだろう.
なおこれらの機能については, Word においても利用可能である. しかし, 煩雑なGUIメニューの操作を要求されるため, 文書構造を指定するだけでも一苦労であり, 太字やフォントサイズの指定など,
アクセスしやすい初期のツールボックスに既定で表示されている, 装飾に関する項目ばかりを弄りがちである.
本書では, \LaTeX の短所たる「とっつきづらさ」を少しでも軽減すべく作成したテンプレートファイルについて, その活用方法を解説していく.

なお, 本書もこのテンプレートに付属するtexファイルを使用してコンパイルされている.
\footnote{
    本テンプレートは原則MITライセンスの下に配布されるが, 本文書については通常の著作権法の下保護されるものとする点に注意されたい.
    これは本記述がなされているdummytext.texファイル, およびそれを含んだ状態でコンパイルすることにより生成されたroot.pdfのみに適用されるものとする.
    テンプレートとしての運用時はこれらを削除して利用すること. (dummytext.tex内にも同様の記述を行ったため, そちらも確認されたい.)
}
拙学ゆえ誤り, 非効率, 非推奨といった要素を含む記法を併用している可能性も大いにあるが, 記述の際の参考にされたい.
